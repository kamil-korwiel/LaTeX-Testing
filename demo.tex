\documentclass{article}
\usepackage{pdflscape}
\usepackage{caption} % Für Tabellen- und Abbildungsüberschriften
\usepackage{array} % Verbesserte Tabellenumgebung
\usepackage{booktabs} % Für schönere Tabellenlinien
\usepackage{fancyhdr} % For controlling headers/footers
\usepackage{lipsum}   % For dummy text
\usepackage{changepage} % Allows margin changes on a single page
\usepackage{geometry}

\begin{document}

\lipsum[1] % Some dummy text

\begin{landscape} % pdflscape sorgt für korrekte Rotation im PDF-Viewer
    \def\maxlenth{0.80in} % Define a variable
        \begin{table}
            \centering
            \newgeometry{margin=0mm} % Remove margins only for this page
            \thispagestyle{empty} % No headers/footers
           
            \caption{Beispiel für eine Tabelle im Querformat} % Tabellenüberschrift
            \label{tab:example_table} % Label für Referenzierung
            
            \begin{tabular}{|p{\maxlenth}|p{\maxlenth}|p{\maxlenth}|p{\maxlenth}|p{\maxlenth}|p{\maxlenth}|p{\maxlenth}|p{\maxlenth}|p{\maxlenth}|}
                \hline
                XXX XXXXXXXXX & XXXX XXXXXX XXXXX & XXXXXXXX XXXXX XXXXXXX XXXXX & XXXXXXXX XXXXXXX X XXXXX XXXX & XXXXX XXXXX XXX XXXXXX XXXXXX & Spalte & Spalte & Spalte & Spalte \\ \hline
                Daten 1  & Daten 2  & Daten 3  & Daten 4  & Spalte & Spalte & Spalte & Spalte & Spalte \\ \hline
                Daten 5  & Daten 6  & Daten 7  & Daten 8  & Spalte & Spalte & Spalte & Spalte & Spalte \\ \hline
                Daten 9  & Daten 10 & Daten 11 & Daten 12 & Spalte & Spalte & Spalte & Spalte & Spalte \\ \hline
            \end{tabular}
            \vspace{0.5cm} % Abstand zur Quellenangabe
            \caption*{\textit{Quelle: Eigene Darstellung oder Quelle XYZ.}} % Quellenangabe unter der Tabelle
        \end{table}
        \restoregeometry % Restore original margins after this page
\end{landscape}

\lipsum[2] % More dummy text

\end{document}